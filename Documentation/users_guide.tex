% !TEX TS-program = pdflatex
% !TEX encoding = UTF-8 Unicode

% Example of the Memoir class, an alternative to the default LaTeX classes such as article and book, with many added features built into the class itself.

%\documentclass[12pt,a4paper]{memoir} % for a long document
\documentclass[12pt,letterpaper]{memoir} % for a short document

\usepackage[utf8]{inputenc} % set input encoding to utf8

\usepackage{colortbl}
\usepackage[dvipsnames]{xcolor}
\usepackage{minted}
\usepackage[section]{placeins}
\usepackage{mdframed}
\usepackage{geometry} % to change the page dimensions
\geometry{letterpaper} % or letterpaper (US) or a5paper or....
\geometry{margin=0.75in} % for example, change the margins to 2 inches all round

% Don't forget to read the Memoir manual: memman.pdf

%%% Examples of Memoir customization
%%% enable, disable or adjust these as desired

%%% PAGE DIMENSIONS
% Set up the paper to be as close as possible to both A4 & letter:
%\settrimmedsize{11in}{210mm}{*} % letter = 11in tall; a4 = 210mm wide
%\setlength{\trimtop}{0pt}
%\setlength{\trimedge}{\stockwidth}
%\addtolength{\trimedge}{-\paperwidth}
%\settypeblocksize{*}{\lxvchars}{1.618} % we want to the text block to have golden proportionals
%\setulmargins{50pt}{*}{*} % 50pt upper margins
%\setlrmargins{*}{*}{1.618} % golden ratio again for left/right margins
%\setheaderspaces{*}{*}{1.618}
%\checkandfixthelayout 
% This is from memman.pdf

%%% \maketitle CUSTOMISATION
% For more than trivial changes, you may as well do it yourself in a titlepage environment
%\pretitle{\begin{center}\sffamily\huge\MakeUppercase}
\posttitle{\par\end{center}\vskip 0.5em}

%%% ToC (table of contents) APPEARANCE
\maxtocdepth{subsection} % include subsections
\renewcommand{\cftchapterpagefont}{}
\renewcommand{\cftchapterfont}{}     % no bold!

%%% HEADERS & FOOTERS
\pagestyle{ruled} % try also: empty , plain , headings , ruled , Ruled , companion

%%% CHAPTERS
\chapterstyle{hangnum} % try also: default , section , hangnum , companion , article, demo

%\renewcommand{\chaptitlefont}{\Huge\sffamily\raggedright} % set sans serif chapter title font
%\renewcommand{\chapnumfont}{\Huge\sffamily\raggedright} % set sans serif chapter number font

%%% SECTIONS
%\hangsecnum % hang the section numbers into the margin to match \chapterstyle{hangnum}
%\maxsecnumdepth{subsection} % number subsections

%\setsecheadstyle{\Large\sffamily\raggedright} % set sans serif section font
%\setsubsecheadstyle{\large\sffamily\raggedright} % set sans serif subsection font

%% END Memoir customization

\title{Chandra Library User's Guide}
\author{Martin Jay McKee}
%\date{} % Delete this line to display the current date

\colorlet{binarybg}{SeaGreen!20!White!80!}
\colorlet{textbg}{Cyan!10!White!90!}
\newcolumntype{g}{>{\columncolor[gray]{0.95}}c}
\newcolumntype{B}{>{\columncolor{binarybg}}c}
\newcolumntype{T}{>{\columncolor{textbg}}c}

%%% BEGIN DOCUMENT
\begin{document}
%\begin{titlepage}
%\center{\Huge{Chandra Library User's Guide}}
%\end{titlepage}

\maketitle
\tableofcontents* % the asterisk means that the contents itself isn't put into the ToC

\chapter{Overview}
Chandra is a collection of almost entirely header-only libraries that are designed to support the development of real-time embedded control systems on microcontroller systems.  While Chandra supports running inside an RTOS, it does not require it nor is it specifically designed for it as the libraries are designed in such a way that functionality can easilby be called from an explicit event loop.  There is a single ``core'' .cpp file that defines the whole Chandra runtime\footnote{Currently there is only a single variable definition which could also be defined directly in the application code.  In future, however, additional values could be defined and, as such, it is cleaner to place them in the .cpp file.}.

\section{Description}
As they stand, the Chandra libraries consist of four sub-libraries: Chandra-Core, Chandra-HAL, Chandra-Control, and Chandra-Aero.  The Chandra Core libraries are functionality that is required for any of the other libraries but which is - otherwise - stand alone.  The core libraries could be used completely by themselves, though they do not provide enough functionality to implement a complete application because they do not contain features such as input and output.  The HAL libraries provide a hardware abstraction layer for the platform with a consistent API.  This is the only sub-library that requires explicit porting to new platforms\footnote{In future, it may make sense to optimized some of the core functionlaity based on the platform.  This might include chosing loop variable sizes or other microoptimizations.  The core functionality should work without requireing any porting, however.}.  The HAL includes things such as a system clock implementation, timing functionality, GPIO, serial peripherals, etc.  Some of the functionality in the HAL is platform independent as it is based directly on lower level functionality.  The porting documentation later in this guide describes what types and functions need to be implemented.  The third sub-library is Chandra-Control which provides observers, filters and control loop implementations.  Finally, Chandra-Aero provides special-purpose functions for aerodynamics related programs.  This includes functionality such as an atmosphere (~altitude, pressure, temperature, density~) model, basic air properties, lift and drag calculation, etc.

\section{Dependencies}
The Chandra libraries are written against the C++14 standard and require a compiler which is compatible with the standard, including language features such as: ranged for loops, generalized lambda expressions, constexpr expressions, and variadic templates.  Any C++14 compliant compiler should have no issues with the code.  Only minimal standard library support is needed but includes: chrono, type\_traits, number\_traits, and std::common\_type.  

To use the code generators bundled with Chandra, it is also necessary to have a Python 3 interpreter installed along with several libraries.  To run the generator for Chandra Units, the Mako Templating engine is required.  For the Kalman Filter Optimizer, Mako and SymPy are required.

The documentation is built in LaTex and requires a number of packages including pygments (~which - in return - requires Python~) and graphicx.  In general, users should not need to rebuild the documentation as Chandra is distributed with the final built PDF of the documentation.
 
\section{Platform Indepedence}
The bulk of the Chandra libraries are platform independent.  The only portion of the libraries which are platform dependent are those within Chandra-HAL -- the \emph{H}ardware \emph{A}bstraction \emph{L}ayer.  Porting of the HAL functionality to new platforms is required but the remainder of the libraries can be used with no changes so long as a C++14 compiler and standard library are available.  If some standard library features are unavaliable on the target platform, they can also be included externally.

\section{User's Guide Organization}
This guide is arranged in five major sections.  The first, this overview, is simply a highlevel overview which should be sufficient for a new user to understand what the Chandra library is - in a basic way - but ability to use the library should not be expected by this point.  The following four sections are intended to provide a deeper introduction to the components in the library, as well as rationale, usage information, and general reference.  The second section of the guide is a getting started chapter which inlcudes ?? complete applications (~targeted at an NXP LPCXpresso1549 development board) to demonstrate basic use of the library, features, code structure, and compilation.  The third section of this guide covers the overall library architecture.  It introduces all the major components of the library and the rationale for including them.  Following the architecture documentation, the is an in depth API (~\emph{A}pplication \emph{P}rogrammer \emph{I}nterface~) reference which describes the complete interface for every type and function.  The final section of the guide is a code ``cookbook'' which includes code fragments to solve specific problems.  This final section is intended to simplify the learning of the library as well as demonstrating some of the less obvious features.

Beyond the primary descriptive sections

\part{Library Introduction}
\chapter{Getting Started}
The Chandra libraries were originally written for personal projects targeting the NXP LPC series of ARM Cortex-M microcontrollers.  Despite this, however, the Chandra libraries were writen from the very beginning to expose an interface that can be ported to multiple architectures without loss of functionality.  The Chandra libraries are designed to e extremely powerful without being too excessively difficult to use.  While they may not be as simple to use as an Arduino compatible platform, the libraries are designed in such a way as to provide more direct access to the hardware and so that they can be optimized\footnote{Compilation of an application built on the Chandra Libraries without compiler optimization \emph{WILL} lead to large slow programs.  Chandra is implemented using templates and other advanced C++ features that cause bloated unoptimized code.  Nevertheless, when compiler optimization is enabled, much of the functionality is optimized to very nearly the hand-written equivelent.  Do not compile Chandra programs without optimization and expect svelt code.  You won't get it.  With optimization, however, you will.} to a much higher level than some of the core Arduino functionality.  Having said that, Chandra is not a competitor to Arduino.  It is aimed at implementing control systems for robotics.  Specifically, it is designed for things like rovers and flight computers.  Additionally, while it could certainly be used for high-level systems, it is mostly intended for deeply embedded, real-time systems.

This section will introduce some of the central concepts and functions in the Chandra library in the context of example applications.  

\section{}
\subsection{}

\chapter{Library Architecture}
\section{Overview}
The following sections provide an outline of the library facilities.  Each feature will be explained and a very basic example of use provided.  Complete description of the API is saved for the API chapter which is arranged in a manner that mirrors this chapter.  Essentially, this chapter is an introduction to functionality and the rationale behind it while the API reference gets into the nuts and bolts of each function, class, and datatype provided by the library.

\section{Core}
\subsection{Math}
\subsubsection{Matrix Types}
The 
ADD SYMMETRIC AND SKEW-SYMMETRIC MATRICIES.  THESE CAN HAVE OPTIMIED STORAGE AND MAY BE POSSIBLE TO OPTIMIZE OPERATIONS FOR THESE SPECIFIC FORMS (I.E. ADDITION CAN BE OPTIMIZED, MULTIPLICATION THAT CREATES A SKEW-SYMMETRIC MATRIC MAY BE OPTIMIZABLE, ETC.

\subsubsection{Matrix Operations}
IT MAY MAKE SENSE TO USE EXPRESSION TEMPLATES TO SELECT THE IMPLEMENTATION OF OPERATIONS.

\subsection{Units}

\section{HAL}
\subsection{Chandra::Chrono}
\subsubsection{System Clock}
To configure the timestamp clock, a preprocessor directive describing the onboard peripheral that will be used as the source. \\
	\begin{centering}
	\begin{tabular}{cgc}
		\rowcolor[gray]{0.8}\textbf{Preprocessor Symbol}  & \textbf{Definition}\\
		SCT\_HARDWARE\_TIMESTAMP\_MODE & Use an SCT clock as the timestamp generator\\
		SYSTICK\_SOFTWARE\_TIMESTAMP\_MODE & Use the SysTick clock as the timestamp generator\\
		
	\end{tabular}
	\end{centering}

\subsubsection{System Frequencies}
\subsubsection{Performance Timers}

\subsection{General Purpose Input/Output}
\subsection{Analog Input/Output}
CURRENTLY THERE IS NO API SPECIFIED FOR DAC, THERE SHOULD BE
\subsection{Communications}
\subsubsection{$I^{2}C$}
\subsubsection{SPI}
\subsubsection{USART}
\subsection{Drivers}
\subsubsection{Barometric Pressure Sensors}
\subsubsection{Inertia Measurement Sensors}
\subsubsection{Servo Drivers}

\section{Control}
\subsection{Kalman Filtering}
The Kalman filtering components of the Chandra library are designed to support multiple approaches to constructing a kalman filter.  Explicit implementation of a Kalman filter directly in source code by configuring the various matricies is simple.  This also has the advantage that it allows for dynamic modification of the matricies during runtime.
\subsubsection{Code Generator}

\subsection{PID}

\subsection{Builtin Estimators}
\subsubsection{Madgwick's AHRS}

\section{Aero}
ATMOSPHERIC PROPERTIES, MODEL OF ATMOSPHERE (PRESSURE, DENSITY, TEMPERATURE, ALTITUDE, ETC.), BASIC AERODYNAMICS

\part{Library Deep Dive}
\chapter{API}

\section{}
\subsection{}

\chapter{Chandra Cookbook}
\section{Clock and Timing}
\subsection{Performance Timers}
\subsubsection{Timing a function}
\subsubsection{Conditional timing of a function}
\subsubsection{Loop frequency measurement}
\subsubsection{Loop frequency statistics}

\section{Digital I/O}
\subsection{Reading a digital pin}
\subsection{Writing to a digital pin}
\subsection{Reading a mechanical switch/button}
\subsection{Manual control of an LED}
\subsection{Automatic PWM of an LED}
\subsection{Automatic PWM of an LED with asynchronous update}
\subsection{Event on a pin falling (rising) edge}
\subsection{Reading a Quadrature Encoder}

\section{Analog I/O}
\subsection{Raw reading of a single ADC channel}
\subsection{Scaled reading of a single ADC channel}
\subsection{Reading multiple ADC channels}

\section{Math}
\subsection{Construct a matrix}
\subsection{Solve a system of equations}
\subsection{Invert a matrix}

\appendix
\part{Appendicies}

\chapter{Setting up MCUXpresso for LPC Series Microcontrollers}
\section{Installation}
\section{Optimization and Standards}
\section{Include Paths}
\section{Preprocessor Definitions}

\chapter{Porting Guide}
The only section of the Chandra libraries that require porting when moving to new platforms is the Chandra-HAL components.  Any other components (core, control, and aero) are platform independent.

ALL COMPONENTS THAT REQUIRE PORTING NEED TO HAVE HEADER TEMPLATES.
\section{Low-Level Register Access}
THIS IS THINGS LIKE CLOCK POWER, PERIPHERAL RESET, GPIO MAPPING, ETC.
\section{Clock and Frequencies}
\section{GPIO}
\section{ADC}
\section{Timers}
THERE IS NO TIMER API DEFINED AT THE MOMENT, BUT THERE REALLY SHOULD BE ONE...
\section{$I^2C$}
\section{SPI}
\section{USART}

\section{(Optional) DAC}
\section{(Optional) Special Peripherals}

\end{document}
