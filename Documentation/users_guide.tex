% !TEX TS-program = pdflatex
% !TEX encoding = UTF-8 Unicode

% Example of the Memoir class, an alternative to the default LaTeX classes such as article and book, with many added features built into the class itself.

%\documentclass[12pt,a4paper]{memoir} % for a long document
\documentclass[12pt,a4paper,article]{memoir} % for a short document

\usepackage[utf8]{inputenc} % set input encoding to utf8

% Don't forget to read the Memoir manual: memman.pdf

%%% Examples of Memoir customization
%%% enable, disable or adjust these as desired

%%% PAGE DIMENSIONS
% Set up the paper to be as close as possible to both A4 & letter:
\settrimmedsize{11in}{210mm}{*} % letter = 11in tall; a4 = 210mm wide
\setlength{\trimtop}{0pt}
\setlength{\trimedge}{\stockwidth}
\addtolength{\trimedge}{-\paperwidth}
\settypeblocksize{*}{\lxvchars}{1.618} % we want to the text block to have golden proportionals
\setulmargins{50pt}{*}{*} % 50pt upper margins
\setlrmargins{*}{*}{1.618} % golden ratio again for left/right margins
\setheaderspaces{*}{*}{1.618}
\checkandfixthelayout 
% This is from memman.pdf

%%% \maketitle CUSTOMISATION
% For more than trivial changes, you may as well do it yourself in a titlepage environment
%\pretitle{\begin{center}\sffamily\huge\MakeUppercase}
\posttitle{\par\end{center}\vskip 0.5em}

%%% ToC (table of contents) APPEARANCE
\maxtocdepth{subsection} % include subsections
\renewcommand{\cftchapterpagefont}{}
\renewcommand{\cftchapterfont}{}     % no bold!

%%% HEADERS & FOOTERS
\pagestyle{ruled} % try also: empty , plain , headings , ruled , Ruled , companion

%%% CHAPTERS
\chapterstyle{hangnum} % try also: default , section , hangnum , companion , article, demo

%\renewcommand{\chaptitlefont}{\Huge\sffamily\raggedright} % set sans serif chapter title font
%\renewcommand{\chapnumfont}{\Huge\sffamily\raggedright} % set sans serif chapter number font

%%% SECTIONS
%\hangsecnum % hang the section numbers into the margin to match \chapterstyle{hangnum}
%\maxsecnumdepth{subsection} % number subsections

%\setsecheadstyle{\Large\sffamily\raggedright} % set sans serif section font
%\setsubsecheadstyle{\large\sffamily\raggedright} % set sans serif subsection font

%% END Memoir customization

\title{Chandra Library User's Guide}
\author{Martin Jay McKee}
%\date{} % Delete this line to display the current date

%%% BEGIN DOCUMENT
\begin{document}
%\begin{titlepage}
%\center{\Huge{Chandra Library User's Guide}}
%\end{titlepage}

\maketitle
\tableofcontents* % the asterisk means that the contents itself isn't put into the ToC

\chapter{Overview}
\section{Description}
\section{Dependencies}
THERE ARE CODE GENERATORS WRITTEN IN PYTHON
\section{Platform Indepedence}

\chapter{Getting Started}
THIS GETTING STARTED CHAPTER SHOULD INCLUDE A COUPLE OF COMPLETE APPLICATIONS.

\section{}
\subsection{}

\chapter{Library Architecture}
\section{Overview}
\section{Core}
\subsection{Math}
ADD SYMMETRIC AND SKEW-SYMMETRIC MATRICIES.  THESE CAN HAVE OPTIMIED STORAGE AND MAY BE POSSIBLE TO OPTIMIZE OPERATIONS FOR THESE SPECIFIC FORMS (I.E. ADDITION CAN BE OPTIMIZED, MULTIPLICATION THAT CREATES A SKEW-SYMMETRIC MATRIC MAY BE OPTIMIZABLE, ETC.

IT MAY MAKE SENSE TO USE EXPRESSION TEMPLATES TO SELECT THE IMPLEMENTATION OF OPERATIONS.

\section{HAL}
\section{Control}
\subsection{Kalman Filtering}
The Kalman filtering components of the Chandra library are designed to support multiple approaches to constructing a kalman filter.  Explicit implementation of a Kalman filter directly in source code by configuring the various matricies is simple.  This also has the advantage that it allows for dynamic modification of the matricies during runtime.  THERE SHOULD ALSO BE A CODE GENERATOR FOR CREATION OF 
\subsection{PID}

\section{Aero}

\chapter{API}
\section{}
\subsection{}

\chapter{Chandra Cookbook}
\section{}
\subsection{}

\chapter{Porting Guide}

\end{document}
